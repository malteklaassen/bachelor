\documentclass[handout]{beamer}
%\usepackage[german]{babel}
%\usepackage[utf8x]{inputenc}
\usepackage[ngerman]{babel}
\usepackage{amsmath}
\usepackage{listings}
\usepackage{tikz}
\usepackage{booktabs}
\lstset{
	frame=single,
	breaklines=true
}

\usetikzlibrary{arrows,automata}


\definecolor{verylightgray}{rgb}{0.9,0.9,0.9}

\usetheme{Madrid}
% Other valid themes
%   Antibes, Bergen, Berkeley, Berlin, Copenhagen
%   Darmstadt, Dresden, Frankfurt, Goettingen, Hannover
%   Ilmenau, JuanLesPins, Luebeck, Madrid, Malmoe
%   Marburg, Montpellier, PaloAlto, Pittsburgh, Rochester
%   Singapore, Szeged, Warsaw, boxes, default

%möglich: Antibes, Darmstadt, Frankfurt, Madrid, Montpellier, Singapore

\usecolortheme{dove}
% Other valid color schemes
%    albatross, beaver, beetle, crane, dolphin
%    dove, fly, lily, orchid, rose, seagull
%    seahorse, whale and the one and only wolverine

%möglich: albatross, beaver, dove, whale

\title[Checkpoint/Restore in Fuzzing]{Checkpoint/Restore in Fuzzing}
\subtitle{Begleitseminar Bachelorarbeit}
\author[Klaassen]{Malte Klaassen}
%\institute[Kurzform]{Institut}
\date{2018-12-14}

\begin{document}

\begin{frame}%1
\titlepage
\end{frame}

\begin{frame}%2
	\frametitle{Inhaltsverzeichnis}
	\tableofcontents%[pausesections]
\end{frame}

\section{Fuzzing}
\begin{frame}
    \frametitle{Fuzzing}
    \begin{itemize}
        \item Testen mit (zufälligen) computergenerierten Inputs
        \item insb.\ zum Abdecken von Grenzfällen etc.
        \item Überwachung der Ausführung auf unerwünschtes Verhalten
        \item Verschiedene Strategien, bspw.\ zur Erzeugung der Inputs
    \end{itemize}
\end{frame}

\begin{frame}
    \frametitle{Fuzzing: Probleme/Beschränkungen}
    \begin{itemize}
        \item Effektivität von Fuzzing beruht auf hohem Durchsatz
        \item Zerlegung komplexer Anwendungen in simple Subsysteme \begin{itemize}
                \item Zusätzlicher Aufwand für mehr Fuzz-Targets
                \item Benötigt Kenntnisse über Struktur der Anwendung
        \end{itemize}
    \end{itemize}
\end{frame}

\begin{frame}
    \frametitle{Fuzzing: Fuzzing-Engines}
    \begin{itemize}
        \item libFuzzer
            \begin{itemize}
                \item Ziel: Leicht zu nutzende, mächtige Fuzzing-Engine
                \item Integration mit LLVM/clang
                \item Nutzt clang-Compilerfeatures (Code-Coverage, Sanitizer, Fuzzing-Engine)
            \end{itemize}
        \item afl
            \begin{itemize}
                \item Ziel: Mächtige, effiziente Fuzzing-Engine
                \item Nutzt einige Compilerfeatures (insb. Code-Coverage, einige Sanitizer)
                \item Benötigt eigene Fuzzing-Binary, spezielle Compiler-Versionen, \dots
            \end{itemize}
    \end{itemize}
\end{frame}

\section{Checkpoint/Restore}
\begin{frame}
    \frametitle{Checkpoint/Restore}
    \begin{itemize}
        \item C/R : Speichern des Zustandes einer Anwendung zur späteren Wiederherstellung
        \item Debugging, Laufzeitoptimierung, Lastoptimierung, Migration
        \item Userspace vs. Kernel vs. Container
    \end{itemize}
\end{frame}

\begin{frame}
    \frametitle{C/R: Userspace}
    \begin{itemize}
        \item Checkpointen einer Anwendung durch eine Userspace-Anwendungen
        \item Sammeln/Wiederherstellen der nötigen Informationen durch:
            \begin{itemize}
                \item Lesen in bspw.\ \texttt{/proc/}
                \item Intercept von System- und Library-Calls (bspw.\ DMTCP)
                \item Nutzung von OS-Features bspw.\ für Speicherinhalte (BLCR, CRIU)
            \end{itemize}
    \end{itemize}
\end{frame}

\begin{frame}
    \frametitle{C/R: Kernel}
    \begin{itemize}
        \item Checkpointen einer Anwendungen durch den Kernel
        \item Kernel hat direkten Zugriff auf alle benötigten Informationen
        \item Benötigt Custom Kernel (Legacy OpenVZ C/R) oder Kernelpatches (Linux-CR)
    \end{itemize}
\end{frame}

\begin{frame}
    \frametitle{C/R: Container}
    \begin{itemize}
        \item Container oder Virtuelle Maschinen als zu checkpointene Anwendung
        \item Integration existierender C/R-Tools in Containerverwaltungssystem (Docker: CRIU)
    \end{itemize}
\end{frame}

\begin{frame}
    \frametitle{C/R-Tools: Capabilities}
    \begin{itemize}
        \item Verschiedene C/R-Tools können verschiedene Features checkpointen und wiederherstellen
            \begin{itemize}
                \item Threads?
                \item Mehrere Prozesse?
                \item Netzwerksockets? In welchen Zuständen?
                \item PIDs? Namespaces?
            \end{itemize}
        \item C/R-Tools arbeiten primär auf Prozessen, nicht Threads
    \end{itemize}
\end{frame}

\section{C/R-Fuzzing: Ansätze}

\begin{frame}
    \frametitle{C/R-Fuzzing: Ansätze}
    \begin{itemize}
        \item Ziel: Lösung von Fuzzing-Problemen mittels C/R-Mechanismen
            \begin{itemize}
                \item Laufzeitoverhead durch wiederholte Setups
                \item Notwendigkeit der Zerlegung in Subsysteme
            \end{itemize}
        \item Frage: Wo wird C/R-Funktionalität implementiert?
            \begin{itemize}
                \item Im Fuzzer, transparent gegenüber dem Fuzz-Target
                \item Im Fuzz-Target, ohne C/R-Support des Fuzzers
            \end{itemize}
        \item Frage: Was für C/R-Funktionalität?
    \end{itemize}
\end{frame}

\begin{frame}
    \frametitle{C/R-Fuzzing: Bisherige Ansätze}
    \begin{itemize}
        \item afl Forkserver
            \begin{itemize}
                \item Einfrieren des Fuzz-Target-Prozesses nach \texttt{execve}, Linking, Initialisierung (oder noch später mit \texttt{\_\_AFL\_INIT();})
                \item Fuzzing findet auf Copy-on-Write-fork dieses Prozesses statt
                \item Forkserver muss vor Nutzung von Childprocesses, Threads, Filedeskriptoren, \dots aufgesetzt werden
                    $\rightarrow$ Reduktion des Setupoverheads aber nur bei simplem Overhead
            \end{itemize}
        \item Mit mächtigen C/R-Tools: Keine (bekannt)
    \end{itemize}
\end{frame}

\begin{frame}
    \frametitle{C/R-Fuzzing: "Naiver" C/R-Ansatz}
    \begin{itemize}
        \item Ähnlich zu afl Forkserver
        \item Zerlege Fuzz-Target in Setup und eigentlichen Test
        \item Führe das Setup einmal aus, checkpointe den Prozess, steige für weitere Ausführungen an dem Checkpoint wieder ein
        \item Implementiert durch den Fuzzer mit entsprechendem Breakpoint in Fuzz-Target oder implementiert im Fuzz-Target selbst
        \item Laufzeitgewinn pro Iteration:
            \begin{equation}
                \begin{split}
                    \frac{T_{Non-C/R}(n) - T_{C/R}(n)}{n} &= \frac{(n-1) T_{Setup} - T_{Checkpoint} - n T_{Restore}}{n} \\
                    &=_{n \to \infty} T_{Setup} - T_{Restore}
                \end{split}
            \end{equation}
    \end{itemize}
\end{frame}


\begin{frame}
    \frametitle{Naiver C/R-Ansatz: C/R-Fuzzer}
    \begin{itemize}
        \item Mächtigere Variante des afl Forkservers
        \item Inputübergabe muss nach dem Checkpoint passieren
        \item Welche Codesegmente in den Setup-Teil verlagert werden können wird durch das C/R-Tool bedingt
        \item $T_{Restore}$ muss klein sein damit ein Laufzeitgewinn vorliegt
        \item Parallel Fuzzing? Timeouts?
    \end{itemize}
\end{frame}

\begin{frame}
    \frametitle{Naiver C/R-Ansatz: C/R-Fuzz-Target}
    \begin{itemize}
        \item Gleiche Überlegungen wie beim C/R-Fuzzer, zusätzlich:
        \item Kompatibilität von C/R-Tool und Fuzzer muss gewährleistet sein
            \begin{itemize}
                \item Genutzte Sanitizer, Fuzzing von Multi-Process-Anwendungen, Restore-In-Place, \dots
            \end{itemize}
        \item Übergabe des Inputs, Speicherung des Zustands, \dots
    \end{itemize}
\end{frame}

\begin{frame}
    \frametitle{C/R-Fuzzing: Exploration eines Zustandsgraphen}
    \begin{itemize}
        \item Ziel: Effizientes Fuzzing einer komplexen Anwendung mit mehrstufigen Eingaben und wohldefiniertem Zustandsgraphen
            \begin{itemize}
                \item Bspw. Implementierung eines Netzwerkprotokolls
            \end{itemize}
        \item Ansatz: Exploration des Zustandsgraphen durch Anwendung von neuen Inputs auf bereits bekannte Zustände
        \item Checkpointing der Anwendung bei neu gefundenen Zuständen $\rightarrow$ Ausführung mit neuen Inputs kann direkt dort fortgeführt werden
    \end{itemize}
\end{frame}

\begin{frame}
    \frametitle{C/R-Fuzzing: Exploration eines Zustandsgraphen}
    \begin{figure}[h]
        \begin{center}
            \begin{tikzpicture}[->,>=stealth',shorten >=1pt,auto,node distance=2.8cm,
                    semithick]
                \tikzstyle{every state}=[fill=lightgray,draw=none,text=black]
                
                \node[initial,state] (A)                    {$s_0$};
                \node[state]         (B) [below left of=A] {$s_1$};
                \node[state]         (D) [below right of=A] {$s_2$};
                \node[state]         (C) [below of=B] {$s_3$};
                \node[state]         (E) [right of=A]       {$?$};
                \node[state]         (F) [below left of=B]       {$?$};
                \node[state]         (G) [below right of=D]       {$?$};
                \node[state]         (H) [right of=C]       {$?$};
                    
                
                \path (A) edge              node {$i_1$} (B)
                          edge              node {$i_2, i_4$} (D)
                          edge [bend left, dotted]  node {$i*$}  (E)
                      (B) edge              node {$i_3$} (C)
                          edge [bend right, dotted]  node {$i*$}  (F)
                      (C) edge [bend right, dotted] node {$i*$}  (H)
                      (D) edge [bend left, dotted] node {$i*$}  (G);
            \end{tikzpicture}
        \end{center}
        \caption{Zustandsübergangsexploration mit Input $i*$}
    \end{figure}
\end{frame}

\begin{frame}
    \frametitle{C/R-Fuzzing: Exploration eines Zustandsgraphen}
    \begin{itemize}
        \item Woran erkennt man einen Zustand?
        \item Übergabe des Inputs?
        \item Welche Mittel stehen zur geführten Generierungen von Inputs zur Verfügung?
        \item Bei einer Implemetierung des C/R im Fuzz-Target:
            \begin{itemize}
                \item Persistente Speicherung der Zustände
                \item Anwendung der Inputs auf verschiedene Serverzustände?
                \item Wie bei dem naiven Ansatz: Kompatibilitätsprobleme
            \end{itemize}
    \end{itemize}
\end{frame}

\section{CRIU}
\begin{frame}
    \frametitle{CRIU}
    \begin{itemize}
        \item "Checkpoint/Restore in Userspace"
        \item Als Userspace-Ersatz zu OpenVZ-Kernel-CR entwickelt (2011)
        \item Viele unterstützte Features
        \item Wird noch aktiv unterstützt
        \item Nutzt unveränderte Binaries, benötigt keine speziellen Initialisierungen bei Programmstart
        \item Nutzung entweder über RPC/C-API oder Commandline-Tool
    \end{itemize}
\end{frame}

\begin{frame}
    \frametitle{CRIU: Funktionsweise}
    \begin{itemize}
        \item CRIU arbeitet (größtenteils) im Userspace
        \item Nutzt einige Kernelfeatures und privilegierte Operationen (seit 3.11 im Mainline Linux-Kernel)
            \begin{itemize}
                \item ptrace, \texttt{CONFIG\_CHECKPOINT\_RESTORE} u.A. für prctl
                \item \texttt{CONFIG\_NAMESPACES} sowie weitere Namespace-Features
                \item Socketmonitoring
            \end{itemize}
    \end{itemize}
\end{frame}

\begin{frame}
    \frametitle{CRIU: Funktionsweise - Checkpointing}
    \begin{itemize}
        \item Einfrieren des Prozessbaumes (freezer cgroup oder ptrace)
        \item Extraktion der Prozessinformationen
            \begin{itemize}
                \item Extern durch Lesen von \texttt{/proc/} und ptrace
                \item Intern durch Injektion eines Parasite-Blobs mittels ptrace
            \end{itemize}
        \item Schreiben des Images, Auftauen oder Beenden der Anwendung
    \end{itemize}
\end{frame}

\begin{frame}
    \frametitle{CRIU: Funktionsweise - Restore}
    \begin{itemize}
        \item Zerlegung des Images in einzelne Prozesse und Zuweisung von Shared Ressources
        \item Erstellung eines entsprechenden Prozessbaumes durch Forken der CRIU-Anwendung
        \item Wiederherstellen der Prozessinformationen
            \begin{itemize}
                \item Extern durch die CRIU-Anwendung (bspw. Speicherinhalte, Sockets, Namespaces)
                \item Intern durch die Nutzung eines Restorer-Blobs in den geforkten Prozessen
            \end{itemize}
        \item Unmapping des Restorer-Contextes, Fortsetzen der Anwendung
    \end{itemize}
\end{frame}

\begin{frame}
    \frametitle{CRIU: CRIU+Fuzzing?}
    \begin{itemize}
        \item CRIU wurde nicht gezielt für Fuzzing entwickelt
        \item Es ergeben sich einige (lösbare) Probleme:
            \begin{itemize}
                \item Wiederholtes Wiederherstellen von Anwendungen mit etablierter TCP-Verbindung $\rightarrow$ Firewall-Regel
                \item Genutzte PID bereits wieder neu vergeben $\rightarrow$ Isolation, bspw.\ mit PID-Namespace
                \item TCP-Timeouts $\rightarrow$ Vergrößerung des Fensters, Manipulation des Images
            \end{itemize}
        \item Performance?
    \end{itemize}
\end{frame}

\begin{frame}
    \frametitle{CRIU: Performance}
    \begin{itemize}
        \item Performance, insb. $T_{Restore}$ von hoher Relevanz für C/R-Fuzzing
    \end{itemize}

\begin{tabular}{| l | l  l  l  l  l | l |}
    \toprule
    Iterationen & $1$byte & $10^2$byte & $10^4$byte & $10^6$byte & $10^8$byte & TCP \\
    \midrule
    10 & 0.292s & 0.290s & 0.292s & 0.301s & 0.627s & 0.982s \\
    50 & 1.269s & 1.269s & 1.256s & 1.298s & 2.332s & 4.615s \\
    100 & 2.459s & 2.494s & 2.477s & 2.558s & 4.482s & 9.170s \\
    500 & 12.205s & 12.228s & 12.250s & 12.648s & 21.564s & 45.541s \\
    \midrule
    $T_{Restore}$ & 0.024s & 0.024s & 0.024s & 0.025s & 0.043s & 0.091s \\
    \bottomrule
\end{tabular}
    \begin{itemize}
        \item insb. $T_{Restore}$ $\geq 24ms$ (auf diesem Laptop)
        \item Linux-CR erreicht Restore-Zeiten von $\leq 1ms$
    \end{itemize}

\end{frame}

\section{C/R-Fuzzing: Implementierung}
\begin{frame}
    \frametitle{C/R-Fuzzing: Implementierungsversuche}
    \begin{itemize}
        \item In Ermangelung mächtiger C/R-Fuzzer: Können wir C/R-Fuzz-Targets implementieren?
        \item Betrachten libFuzzer/afl und CRIU
        \item Versuch der Implemetierung des Naiven C/R-Fuzzing-Ansatzes
        \item Unerfolgreich, aufgrund einer Reihe von Problemen
    \end{itemize}
\end{frame}

\begin{frame}
    \frametitle{libFuzzer + CRIU}
    \begin{itemize}
        \item CRIU ist inkompatibel mit einer Nutzung von ASan
        \item CRIU setzt Wiederherstellung unter gleicher PID vorraus - libFuzzer verhindert dies
        \item libFuzzer besitzt nur extrem eingeschränkten Multi-Process-Support, ein wiederherstellen in-place wäre nötig
    \end{itemize}
\end{frame}

\begin{frame}
    \frametitle{afl + CRIU}
    \begin{itemize}
        \item Da durch den Forkserver Shared Memory mit anderen Prozessen vorliegt scheitert das Checkpointing
        \item Wie bei libFuzzer: Fehlender/eingeschränkter Multi-Process-Support, ASan Inkompatibilität
    \end{itemize}
\end{frame}

\section{Ergebnisse und Ausblick}

\begin{frame}
    \frametitle{Ergebnisse}

\end{frame}

\begin{frame}
    \frametitle{Ausblick}
\end{frame}

\begin{frame}

\end{frame}

\begin{thebibliography}{}
\bibitem[1]{1}
    Lesser-known features of afl-fuzz; Michał Zalewski \url{https://lcamtuf.blogspot.com/2015/05/lesser-known-features-of-afl-fuzz.html}
\end{thebibliography}



\end{document}

%%% OLD %%%

%\subsection{sub}

%\begin{frame}[c]%24
%\frametitle{Vielen Dank für Ihre Aufmerksamkeit}
%\framesubtitle{Untertitel}
%\begin{itemize}
%\item Ein Item
%\end{itemize}
%\end{frame}
